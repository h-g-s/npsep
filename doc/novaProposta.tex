%% LyX 2.1.1 created this file.  For more info, see http://www.lyx.org/.
%% Do not edit unless you really know what you are doing.
\documentclass[english]{article}
\usepackage[T1]{fontenc}
\usepackage[latin9]{inputenc}

\makeatletter

%%%%%%%%%%%%%%%%%%%%%%%%%%%%%% LyX specific LaTeX commands.

\newcommand*\LyXZeroWidthSpace{\hspace{0pt}}

\makeatother

\usepackage{babel}
\begin{document}
The conflict graph constructed by our approach consider a vertex for
the bounds (lower and upper bounds) of each variable. There is an
edge between two vertices when at most one of the bounds represented
by the vertices can be set as the value of the corresponding variable,
in a feasible solution. Trivially, there is a edge between the lower
and the upper bound for each variable.

We consider problems whose variables may assume only non-negative
integer values\LyXZeroWidthSpace \LyXZeroWidthSpace .

Let $i$ the $i$-th row of $A$. Suppose that $i$ was ordered in
non-decreasing order according to its coefficients. Namely:

$a_{ij}\leq a_{ij+1},\forall j\in\{1..n-1\}$


\subsection*{Clique discover - C1}

Given a constraint $i$ with the following form:

$\sum_{j\in\{1..n\}}a_{ij}\, x_{j}\{=,\leq\}b_{i}$

Suppose we are analysing two particular variables $x_{\hat{j}}$ and
$x{}_{\hat{j}+1}$ with respect to this constraint. Let:

$L_{x_{\hat{j}}=u_{\hat{j}},x_{\hat{j}+1}=u_{\hat{j}+1}}^{i}=\sum_{j\in\{1..n\}\setminus\{\hat{j},\hat{j}+1\}}min(0,a_{ij})u_{j}+a_{i\hat{j}}u_{\hat{j}}+a_{i\hat{j}+1}u_{\hat{j}+1}$ 

In this case,$L_{x_{\hat{j}}=u_{\hat{j}},x_{\hat{j}+1}=u_{\hat{j}+1}}^{i}$
is the minimum value of the left-hand side of the constraint $i$,
considering the assignments $x_{\hat{j}}=u_{\hat{j}}$ and $x{}_{\hat{j}+1}=u_{\hat{j}+1}$.
If $L_{x_{\hat{j}}=u_{\hat{j}},x_{\hat{j}+1}=u_{\hat{j}+1}}^{i}>b_{i}+\epsilon$,
there is a clique involving all upper bounds of the variables in the
interval $[k,n]$ whose coefficients, in constraint $i$, are nonzero. 

This approach can be used to avoid analysing constraints which have
no cliques. Let $k$ the index of the highest coefficient in $i$.
Thus, if $L_{x_{\hat{j}}=u_{\hat{j}},x_{\hat{j}+1}=u_{\hat{j}+1}}^{i}\leq b_{i}+\epsilon$,
there are no conflicts in this constraint.


\subsection*{Clique discover - Complement of variables - CC}

Given a constraint $i$ with the following form:

$\sum_{j\in\{1..n\}}a_{ij}\, x_{j}\{=,\geq\}b_{i}$

Suppose we are analysing two particular variables $x_{\hat{j}}$ and
$x{}_{\hat{j}+1}$ with respect to this constraint. Let:

$U_{x_{\hat{j}}=l_{\hat{j}},x_{\hat{j}+1}=l_{\hat{j}+1}}^{i}=\sum_{j\in\{1..n\}\setminus\{\hat{j},\hat{j}+1\}}max(0,a_{ij})u_{j}+a_{i\hat{j}}l_{\hat{j}}+a_{i\hat{j}+1}l_{\hat{j}+1}$ 

In this case,$U_{x_{\hat{j}}=l_{\hat{j}},x_{\hat{j}+1}=l_{\hat{j}+1}}^{i}$
is the maximum value of the left-hand side of the constraint $i$,
considering the assignments $x_{\hat{j}}=l_{\hat{j}}$ and $x{}_{\hat{j}+1}=l_{\hat{j}+1}$.
If $U_{x_{\hat{j}}=l_{\hat{j}},x_{\hat{j}+1}=l_{\hat{j}+1}}^{i}<b_{i}+\epsilon$,
there is a clique involving all lower bounds of the variables in the
interval $[k,n]$ whose coefficients, in constraint $i$, are nonzero. 

This approach can be used to avoid analysing constraints which have
no cliques. Let $k$ the index of the highest coefficient in $i$.
Thus, if $U_{x_{\hat{j}}=l_{\hat{j}},x_{\hat{j}+1}=l_{\hat{j}+1}}^{i}\geq b_{i}+\epsilon$,
there are no conflicts in this constraint.


\subsection*{Pairwise conflict detection}

When the constraint does not satisfy the previously formats the algorithm
uses a pairwise conflict detection mechanism. The method checks for
pairs of variables in this constraint whose activation (or deactivation)
cannot occur at the same time without violating it. Thus, for a constraint,
pairwise conflicts detection can be computed in $O(n^{2})$.

For each pair of variables, for each combination of these variable
values, we compute a new RHS, called $B$, using the old RHS ($b_{i}$)
plus the sum of the most negative (or positive, in case of $\geq$
constraints) value of the other variables. Let $x_{k}$ and $x_{l}$
the variables to be analyzed. The new right-hand side is calculated
as follows:

$B=b_{i}-\sum_{j\in\{1..n\}\setminus\{k,l\}}min(0,aij)u_{j}$ ($B=b_{i}-\sum_{j\in\{1..n\}\setminus\{k,l\}}max(0,aij)u_{j}$)

So, there are four possibilities to detect conflicts:
\begin{enumerate}
\item if $a_{ik}u_{k}+a_{il}u_{l}>B$ ($a_{ik}u_{k}+a_{il}u_{l}<B$): there
is a conflict between upper bound of$x_{k}$ and upper bound of$x_{l}$;
\item if $a_{ik}u_{k}+a_{il}l_{l}>B$ ($a_{ik}u_{k}+a_{il}l_{l}<B$): there
is a conflict between upper bound of $x_{k}$ and lower bound of$x_{l}$;
\item if $a_{ik}l_{k}+a_{il}u_{l}>B$ ($a_{ik}l_{k}+a_{il}u_{l}<B$): there
is a conflict between lower bound of $x_{k}$ and upper bound of$x_{l}$;
\item if $a_{ik}l_{k}+a_{il}l_{l}>B$ ($a_{ik}l_{k}+a_{il}l_{l}<B$): there
is a conflict between lower bound of$x_{k}$ and lower bound of $x_{l}$;\end{enumerate}

\end{document}
